\documentclass{beamer}

% \usepackage{beamerthemesplit} // Activate for custom appearance

%\title{Linear Equations Without Division}
%\author{David Eisenbud\\
%Mathematics}
%
%\date{\today}
\begin{document}
\frame{Who am I?
\begin{itemize}
\item David Eisenbud (Mathematics), PhD University of Chicago 1970
\item Brandeis 1970-1997,  Berkeley since 1997. Meanwhile:
\item Mathematical Sciences Research Institute: Director 1997--2007 and
2013--22
\item Simons Foundation: Director of Math/Physical Science 2010--12 , now Trustee 
\item My project: ``Infinite Free Resolutions'': linear equations with linearly dependent solutions
\end{itemize}}

\frame{
\centerline{$ax +by = 0\quad  \Longrightarrow\quad  y= ax/b.$
\ {\it What if you can't divide?}}
\medbreak
$\bullet$ Algebraic geometry studies a space X by the functions on it. 
If $b$ is a function that takes value 0 somewhere, division is impossible. 
\medbreak

$\bullet$ Example:
\begin{itemize}
\item  $ X =\{(0,0), (1,0),(0,1)\},$ 3 points.
\medbreak
If $a,b$ are the coordinate functions, then
$ab =a(a-1)=b(b-1) = 0$.
\medbreak
\item The solutions of $ax+by=0$ are the 4 columns of:
$\begin{pmatrix}b&a-1&0&0\\0&0&a&b-1 \end{pmatrix}$: 
\medbreak
These are dependent, with 8 linear relations. 
Repeating this, we get exponential growth.
\end{itemize} 
\medbreak
What's the structure?
}
\end{document}
