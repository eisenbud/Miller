\documentclass{beamer}
\usepackage{graphicx}
\usepackage{rotating}
\usepackage{pdflscape}
% \usepackage{beamerthemesplit} // Activate for custom appearance

%\title{Linear Equations Without Division}
%\author{David Eisenbud\\
%Mathematics}
%
%\date{\today}
\begin{document}

\frame{ \centerline{David Eisenbud}
\smallbreak
\centerline{
%\includegraphics[width=.8\textwidth]{"David with dodecahedron balloons.JPG"}
}
%\includegraphics[width=\textwidth]{"David-upsidedown.jpeg"}
%\end{rotate}
%}
%\includegraphics[width=\textwidth]{"David-upsidedown.jpeg"}

My project: find \emph{finite} geometric invariants from some \emph{infinite} algebraic structures
that come from the geometry.
}

%\frame{David Eisenbud
%\begin{itemize}
%\item David Eisenbud (Mathematics), PhD University of Chicago 1970
%\item Brandeis 1970-1997,  Berkeley since 1997. Meanwhile:
%\item Mathematical Sciences Research Institute: Director 1997--2007 and
%2013--22
%\item Simons Foundation: Director of Math/Physical Science 2010--12 , now Trustee 
%\item My project: ``Infinite Free Resolutions'': linear equations with linearly dependent solutions
%\end{itemize}}

\frame{
$\bullet$ The geometry of a space $X$ is reflected in the functions on it and the linear
equations they satisfy. If $b$ is a nonzero number then

\centerline{$ax +by = 0\quad  \Longrightarrow\quad  y= ax/b,$}

\medbreak
but if $b$ is a function that takes value 0 somewhere, division is impossible. What then?
\medbreak

$\bullet$ Example:
\begin{itemize}
\item  $ X =\{(0,0), (1,0),(0,1)\},$ 3 points.
\medbreak
If $a,b$ are the coordinate functions, then
$ab =a(a-1)=b(b-1) = 0$.
\medbreak
\item The solutions of $ax+by=0$ are the 4 columns of:
$\begin{pmatrix}b&a-1&0&0\\0&0&a&b-1 \end{pmatrix}$: 
\medbreak
These are dependent, with 8 linear relations. 
Repeating this, we get exponential growth.
\end{itemize} 
\medbreak
Is there a finite description of such infinite structures?
}
\end{document}
