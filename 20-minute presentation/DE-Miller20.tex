\documentclass[12pt, aspectratio=169]{beamer}
%\usetheme{Dresden}
\usetheme{Copenhagen}
\usecolortheme{beaver}
\def\C{{\mathbb C}}
\setbeamertemplate{navigation symbols}{}
\setbeamertemplate{headline}{}
%\setbeamercovered{transparent}
% \usepackage{beamerthemesplit} // Activate for custom appearance

\title{Riemann surfaces and Weierstrass points}
\author{David Eisenbud}
\date{November 5, 2024}

\begin{document}
\maketitle


\begin{frame}{}
\begin{columns}
\column{.3\textwidth} This story starts with \alert{Bernhard Riemann} (1826--1866; PhD 1851 under Gauss).
Before Riemann, ``curves" were things in the plane, defined by one equation in 2 variables. After Riemann\dots
\column{.7\textwidth}
\begin{figure}
    \centering
    \includegraphics[width=0.7\textwidth]{"Bernhard-Riemann.jpg"}
    \caption{\label{fig:Riemann}MyFig Caption.}
\end{figure}
\end{columns}
 \end{frame}

\begin{frame}{Compact Riemann Surfaces}
\begin{columns}
 \column{.5\textwidth}
\vskip -.4in

 Genus 0: the Riemann Sphere
 \begin{figure}
    \centering
    \includegraphics[width=1.2\textwidth]{"Rsphere2.jpg"}
   % \caption{}
\end{figure}
...with its unique complex structure
%\vskip 1in
%%%%%%
\column{.5\textwidth}
Genus 3: 
\vskip 0in
\begin{figure}{}
    \centering
    \includegraphics[width=.7\textwidth]{"RiemannSurface.png"}
    %\caption{}
\end{figure}
Riemann: there is a 6 = 3g-3 dimensional family of different complex structures. 
\bigskip

How can you tell one from another??
\end{columns}
\end{frame}


\begin{frame}{Weierstrass points and Weierstrass semigroup}
\begin{columns}
 \column{.5\textwidth}
 The only differentiable functions
 everywhere defined on a Riemann surface $X$ are the constants;
 but there are functions that are defined except
 at a point $p\in X$, where there is a pole 
 $$
 z\mapsto \frac{1}{(z-p)^d} + \dots
 $$
 (here of order $d$). At most points $p$ on a surface of
 genus $g$, the possible values of $d$ are
$W(p) = \{0, g+1,g+2,\dots\}$.

\column{.5\textwidth}
But at finitely many \alert{Weierstrass points} $p$
there is a different list $W(p)$ of pole orders. Also, since
$$
\frac{1}{(z-p)^d}\frac{1}{(z-p)^e} = \frac{1}{(z-p)^{d+e}}
$$
the list $W(p)$ is closed under addition, a \alert{semigroup},
called the \alert{Weierstrass semigroup} of $p$.

\bigskip

The Weierstrass points are like lighthouses on $X$.
\end{columns}
 
\end{frame}
 \begin{frame}{A long-running conversation}
 
\begin{columns}
\column{.55\textwidth}
\alert{Karl Weierstrass} (1815--1897)
\begin{figure}
    \flushleft
    \includegraphics[width=.3\textwidth]{"KarlWeierstrassSmall.pdf"}
\end{figure}
  
\begin{small}
 ``A mathematician who is not something of a poet\\ will never be a complete mathematician" 
\end{small}
 \smallskip
 
(1860) If $p$ is a point on a Riemann surface of genus $g$,
 then $W(p)$ is missing precisely $g$ numbers; for example
 $
\{0,g+1,g+2,\dots\},
 $
 or
$
\{0,2, 4,\dots, 2g, 2g+1,\dots\}.
 $
 
 %%%%%%%
\column{.45\textwidth}
\alert{Adolf Hurwitz} (1859--1919)
\begin{figure}
    \flushleft
    \includegraphics[width=.4\textwidth]{"Adolf_Hurwitz.jpg"}
\end{figure}
\begin{small}
 %Question:(1893) 
 \begin{block}{Question:}
Is every semigroup the Weierstrass semigroup of a point
on a Riemann surface?
\end{block}
 \end{small}
\end{columns}

\end{frame}

\begin{frame}{Answers?}
 
\begin{itemize}
 \item<1-6> (Hurwitz's Question, 1893):\alert{ Is every semigroup the Weierstrass semigroup of a point
on a Riemann surface?}
\item <2-> \qquad (Haure, 1896): No!
\item<3-> \qquad\qquad but his proof was wrong.
\item<4-> \qquad Hensel-Landsberg (1901) -- the leading text of the time: Yes!
\item<5-> \qquad\qquad but that proof was wrong, too.
\item<6-> \qquad \alert{Ragnar Buchweitz} (1980): The semigroup generated by
$S = \{0, 13, 14, 15, 16, 17, 18, 20, 22, 23\} $ is NOT a Weierstrass semigroup! 
\smallskip

\alert{Proof:}<7-> Let $G= \hbox{gaps of S}$. Then $\#G = 16$. Suppose that $S$ were the Weierstrass semigroup
of a point on a Riemann surface $X$ with cotangent bundle $\omega_X$. By the Riemann-Roch theorem,
$h^0(\omega_X) = 16$ and $h^0(\omega_X\otimes \omega_X) = 45$; but $\#(G+G) = 46$. This gives a  contradiction.\qed
\end{itemize}
\end{frame}

\begin{frame}{On the positive side}
To prove that a semigroup $S$ does appear as a Weierstrass semigroup on some
Riemann surface $X$, use a ``family'' of
Riemann surfaces  ($=$ smooth projective algebraic curves) $X_u$ ``degenerating'' to a singular curve $X_0$.
 Given the semigroup $S$ you
have to 1) find a singular curve $X_0$ exhibiting some analogue of a Weierstrass point with semigroup $S$;
2) prove the existence of the family $\mathcal X$ with smooth fibers $X_u$ has that Weierstrass point;
3) prove that at least some of the smooth fibers have Weierstrass points of the desired type.

\begin{itemize}
 \item Pinkham (1974): the singular curve is made from the semigroup; it is the set 
of maximal ideals in $\C[\{t^a\}_{a\in S}]$. It has a complicated cusp singularity. 1) is easy, 2) is hard, 3) is easy.

\item E-Harris (1976), Pflueger (2018), \dots with the method of ``limit linear series''; the singular curve is a reducible curve ``of compact type''. 1) is easy; 2) is easy; 3) is hard.
\end{itemize}
\end{frame}

\begin{frame}{Scorecard 2024}\
\begin{itemize}
\item<1-> Assymptotic number of genus $g$ semigroups: $g^\Phi$, where $\Phi = \frac{1+\sqrt 5}{2}\sim 1.618$, the ``Golden ratio'' (Euclid, 300 BCE).
 \item<2-> Existence: 
\begin{itemize}
  \item<3-> E-Harris, Pflueger, \dots, limit series method: all genera, but restricted $S$ (``effective weight'' $\leq g-1$).
\item<3-> E-Schreyer (2024), a variation of Pinkham's method (plus clever computation): all semigroups with $g\leq 10$. Maybe the same methods will decide the question up to $g =15$\dots .
\end{itemize}
\item<4-> Non-existence: Buchweitz's method gives sporadic examples in genus $g \geq 16$.
\item<5-> Answer to Hurwitz's question?
\begin{itemize}
  \item<6-> \alert{Assymptotic percentage of semigroups  known not to be Weierstrass: 0\%.}
 \item<6-> \alert{Assymptotic percentage of semigroups  known to be Weierstrass: 0\%.}
\end{itemize}
 \end{itemize}
\end{frame}

\begin{frame}{Postscript} 
Did you notice that there were no women in the story thus far? That has changed: 
\bigskip

 \alert{Hannah Larson} (Thesis 2023; now Assistant Professor at Berkeley) asks and answers a refined version of Hurwitz's question (catch phrase: Brill-Noether theory over the Hurwitz space). 
 \bigskip
 
We share a student, \alert{Feiyang Lin}--also a woman--is working on a still more refined version.
 
\end{frame}

\begin{frame}

\begin{example}[Pinkham, 1974]
Let $X$ be the plane curve with equation $$y^3-(x^4 -u) = 0;$$ when $u\neq 0$ this is a Riemann surface, but as $u$ goes to 0 it becomes the curve 
defined by the equation $y^3-x^4$. This is parameterized by $$t\mapsto (t^3, t^4),$$ and has a cusp singularity at 0. Pinkham showed that, for small $u$, the
point with coordinates $$y=0, x=u^{1/4}$$ is a Weierstrass point with 
semigroup $\{3,4,5,\dots\}$
\end{example}
\end{frame}
\end{document}

