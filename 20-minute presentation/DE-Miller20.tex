\documentclass[12pt, aspectratio=169]{beamer}
%\usetheme{Dresden}
\usetheme{Copenhagen}
\usecolortheme{beaver}
\def\C{{\mathbb C}}
\setbeamertemplate{navigation symbols}{}
\setbeamertemplate{headline}{}
%\setbeamercovered{transparent}
% \usepackage{beamerthemesplit} // Activate for custom appearance

\title{Riemann surfaces and Weierstrass points}
\author{David Eisenbud}
\date{November 5, 2024}

\begin{document}
\maketitle


\begin{frame}{}
\begin{columns}
\column{.3\textwidth} This story starts with \alert{Bernhard Riemann} (1826--1866; PhD 1851 under Gauss).
Before Riemann, ``curves" were things in the plane, defined by one equation in 2 variables. After Riemann\dots
\column{.7\textwidth}
\begin{figure}
    \centering
    \includegraphics[width=0.7\textwidth]{"Bernhard-Riemann.jpg"}
    %\caption{\label{fig:Riemann}MyFig Caption.}
\end{figure}
\end{columns}
 \end{frame}

\begin{frame}{Compact Riemann Surfaces}
The Riemann surfaces of \alert{genus $g$} look like donuts with $g$ holes; for example:
\begin{columns}
 \column{.5\textwidth}
\vskip -.4in

 Genus 0: the Riemann Sphere
 \begin{figure}
    \centering
    \includegraphics[width=1.2\textwidth]{"Rsphere2.jpg"}
   % \caption{}
\end{figure}
...with its unique complex structure
%\vskip 1in
%%%%%%
\column{.5\textwidth}
Genus 3: 
\vskip 0in
\begin{figure}{}
    \centering
    \includegraphics[width=.7\textwidth]{"RiemannSurface.png"}
    %\caption{}
\end{figure}
Riemann: there is a 6 = 3g-3 dimensional family of different complex structures. 
\medskip

How can you tell one from another??
\end{columns}
\end{frame}


\begin{frame}{Weierstrass points}
\begin{columns}
 \column{.5\textwidth}
 The only differentiable functions
 everywhere defined on a Riemann surface $X$ are the constants;
 but there are functions that are defined except
 at a \alert{single point} $p\in X$, where there is a pole 
 $$
 z\mapsto \frac{1}{(z-p)^d} + \dots
 $$
 (here of order $d$). At most points $p$ on a surface of
 genus $g$, the possible values of $d$ are
$W(p) = \{0, g+1,g+2,\dots\}$.

\column{.5\textwidth}
But at finitely many points $p$
there is a different list $W(p)$ of pole orders. These are called \alert{Weierstrass points}.
The Weierstrass points are landmarks that rigidify on $X$.
\end{columns}
 \end{frame}


\begin{frame}{Numerical semigroups.}
 
 A \alert{semigroup} (more precisely `` numerical semigroup'') is a set of  non-negative integers, 
  \alert{closed under addition}, \alert{containing 0}. and containing all large integers.  For
 example 
 $
\{0,2, 4, 6, 7, 8,\dots\}.
 $
or 
$
 \{0,g+1,g+2,\dots\},
 $
where the 3 dots represent ``all larger numbers''.
\bigskip
\end{frame}


\begin{frame}{The set of pole orders is a numerical semigroup}
A function on a Riemann surface with a pole of order $d$ at $p\in X$ is locally of the form
$
\frac{1}{(x-p)^d} .
$
Since 
$$
\frac{1}{(z-p)^d}\frac{1}{(z-p)^e} = \frac{1}{(z-p)^{d+e}}
$$
 the set of pole orders $W(p)$ is \alert{closed under addition}.  Since the constant function has pole order 0 at every point, $W(p)$ \alert{contains 0}. The Riemann-Roch theorem implies that, if the genus of $X$ is $g$, then
  \alert{every number $\geq 2g$ is in $W(p)$}.
  
 \medskip
 In other words, $W(p)$, the set of pole orders of functions on $X$ that have poles only at $p$, is a \alert{numerical semigroup}.
 
\end{frame}

 \begin{frame}{A long-running conversation}
 
\begin{columns}
\column{.55\textwidth}
\alert{Karl Weierstrass} (1815--1897)
\begin{figure}
    \flushleft
    \includegraphics[width=.3\textwidth]{"KarlWeierstrassSmall.pdf"}
\end{figure}
  
\begin{small}
 ``A mathematician who is not something of a poet\\ will never be a complete mathematician" 
\end{small}
 \smallskip
 
(1860) If $p$ is a point on a Riemann surface of genus $g$,
 then $W(p)$ is missing precisely $g$ numbers, called\alert{gaps}. 
 %%%%%%%
\column{.45\textwidth}
\alert{Adolf Hurwitz} (1859--1919)
\begin{figure}
    \flushleft
    \includegraphics[width=.4\textwidth]{"Adolf_Hurwitz.jpg"}
\end{figure}
\begin{small}
 %Question:(1893) 
 \begin{block}{Question:}
Is every semigroup the Weierstrass semigroup of a point
on a Riemann surface?
\end{block}
 \end{small}
\end{columns}

\end{frame}


\begin{frame}{Answers?}
 
\begin{itemize}
 \item<1-6> (Hurwitz's Question, 1893):\alert{ Is every semigroup the Weierstrass semigroup of a point
on a Riemann surface?}
\item <2-> \qquad (Haure, 1896): No!
\item<3-> \qquad\qquad but his proof was wrong.
\item<4-> \qquad Hensel-Landsberg (1901) -- the leading text of the time: Yes!
\item<5-> \qquad\qquad but that proof was wrong, too.
\item<6-> \qquad \alert{Ragnar Buchweitz} (1980): The semigroup generated by
$S = \{0, 13, 14, 15, 16, 17, 18, 20, 22, 23\} $ is NOT a Weierstrass semigroup! 
\smallskip

\begin{block}{Proof}<7>
 Let $G= \hbox{gaps of S}$. Then $\#G = 16$. Suppose that $S$ were the Weierstrass semigroup
of a point on a Riemann surface $X$ with cotangent bundle $\omega_X$. By the Riemann-Roch theorem,
$h^0(\omega_X) = 16$ and $h^0(\omega_X\otimes \omega_X) = 45$; but $\#(G+G) = 46$. This gives a  contradiction.\qed
\end{block}

\end{itemize}
\end{frame}

\begin{frame}{How to find a Weierstrass point}

Here is a special case of a general method (Pinkham, 1974): For any complex number $u$, let
$X_{u}$ be the plane curve with equation $y^3-(x^4 -u) = 0.$ 
\vskip -.19in
\begin{figure}
  \centering
    \includegraphics[height=1in]{"3curves.pdf"}
\end{figure} 
\vskip -.1in
The figure shows the curves $X_0$, $X_{1}$ and the line $y-1 = 0$.
The complex points of $X_1$ form a Riemann surface of genus 3, but $X_{0}$ has a cusp singularity at 0 (not obvious in this real slice.)
The Riemann surface $X_{1}$ has a Weierstrass point with 
semigroup $\{3,4,6,7,8,\dots\}$ at $x=0, y=-1$. You can ``see'' the function with a pole of order 4: the line
$y-1=0$ is 4-fold tangent to the curve $X_{1}$, so the function $1/(y-1)$ has a pole of order 4
at the point $x=0, y=-1$.
\end{frame}

\begin{frame}{A general method}
To prove that a semigroup $S$ is a Weierstrass semigroup on some
Riemann surface $X_{u}$, use a ``family'' of
Riemann surfaces  ($=$ smooth projective algebraic curves) $X_u$ ``degenerating'' to a singular curve $X_0$.
 Given the semigroup $S$ you
have to
\begin{itemize}
\item find a singular curve $X_0$ exhibiting some analogue of a Weierstrass point with semigroup $S$;
\item prove the existence of the family $\mathcal X$ with smooth fibers $X_u$ has that Weierstrass point;
\item prove that at least some of the smooth fibers have Weierstrass points of the desired type.

\end{itemize}

\end{frame}



\begin{frame}{Two realizations of the method}

\def\C{{\mathbb C}}
\begin{itemize}
 \item Pinkham (1974): the singular curve is made from the semigroup. If the
 semigroup is generated by $a_{1}, \dots a_{s}$ it is the 
 curve in $\C^{s}$ parameterized by
 $$
 t\mapsto (t^{a_{1}},\dots, t^{a_{s}}).
 $$
  It has a complicated cusp singularity. 1) is easy, 2) is hard, 3) is easy.

\item Eisenbud-Harris (1976), Pflueger (2018), \dots with the method of ``limit linear series''; the singular curve is a reducible curve ``of compact type''. 1) is easy; 2) is easy; 3) is hard.
\end{itemize}
\end{frame}

\begin{frame}{Scorecard 2024}\
\begin{itemize}
\item<1-> Assymptotic number of genus $g$ semigroups: $g^\Phi$, where $\Phi = \frac{1+\sqrt 5}{2}\sim 1.618$, the ``Golden ratio'' (Euclid, 300 BCE).
 \item<2-> Existence: 
\begin{itemize}
  \item<3-> Eisenbud-Harris, Pflueger, \dots, limit series method: all genera, but restricted $S$ (``effective weight'' $\leq g-1$).
\item<3-> Eisenbud-Schreyer (2024), a variation of Pinkham's method (plus clever computation): all semigroups with $g\leq 10$. Maybe the same methods will decide the question up to $g =15$\dots .
\end{itemize}
\item<4-> Non-existence: Buchweitz's method gives sporadic examples in genus $g \geq 16$.
\item<5-> Answer to Hurwitz's question?
\begin{itemize}
  \item<6-> \alert{Assymptotic percentage of semigroups  known to be Weierstrass: 0\%.}
 \item<7-> \alert{Assymptotic percentage of semigroups  known {\it\bf not} to be Weierstrass: 0\%.}
\end{itemize}
 \end{itemize}
\end{frame}

\begin{frame}{Postscript} 
Did you notice that there were no women in the story thus far? That has changed: 
\bigskip
\begin{columns}
\column{.4\textwidth}



 \alert{Hannah Larson} (Thesis 2023; now Assistant Professor at Berkeley) asks and answers a refined version of Hurwitz's question (catch phrase: Brill-Noether theory over the Hurwitz space).
 \bigskip
 
We share a student, \alert{Feiyang Lin} who is working on a still more refined question.

\column{.4\textwidth}

\vskip-.1in

\begin{figure}
    %\flushleft
    \includegraphics[width=.4\textwidth]{"HannahLarson.jpg"}
\end{figure}

\vskip -.2in
\begin{figure}
    %\flushleft
    \includegraphics[width=.5\textwidth]{"FeiyangLin.jpg"}
\end{figure}

 \end{columns}

\end{frame}
\end{document}


